\documentclass[12pt,oneside]{sotsuken_paper}

% タイトル
\title{安全な教育用マルチコプターの開発}
\author{北山天斗・剱崎健太郎}

\begin{document}
% 行間
\setlength{\baselineskip}{9truemm}

% 文字間
\kanjiskip=.53zw plus 3pt minus 3pt
\xkanjiskip=.53zw plus 3pt minus 3pt

% 目次
\tableofcontents
%\newpage

% 本文

\appendix
\renewcommand{\thesection}{付録.\ \arabic{section}}
\renewcommand{\thesubsection}{付録.\ \arabic{section}-\arabic{subsection}.}

\section{レーザー加工機}
機体,球体の製作にはレーザー加工機を使用するため,レーザー加工機について説明する.
使用する加工機の機種はGCC LaserPro Spiritである.

\subsection{材料}
加工可能な大きさは彫刻加工で640mm×460mm,カット加工で736mm×460mm,厚さ5mm程度である.

加工可能な材質を以下に示す.

\begin{itemize}
	\item アクリル
	\item 木材
	\item ゴム
	\item ガラス
	\item 皮
	\item 石
	\item コルク
	\item ダンボール
\end{itemize}

加工不可な材質を以下に示す.

\begin{itemize}
	\item 塩ビ系(有毒ガスが発生するため)
	\item 金属系(レーザー光が反射するため)
	\item 鏡(レーザー光が反射するため)
	\item テフロン
	\item プラスチック
\end{itemize}

\subsection{加工データ}
加工データは主に加工機横に設置されているパソコンでCorelDRAWというペイントソフトを用いて作成する.
CorelDRAW以外にもAutoCADなどで作成したdxf形式の2次元CADデータやpng,jpgなどの画像も読み込むことができる.

\subsection{加工時の設定}
加工(印刷)時の環境設定について述べる.
主に設定する必要があるのは以下の項目である.

\begin{itemize}
	\item DPI:1インチの線を何ドットで表現するか(125,250,300,380,500,600,760,1000,1500の中から選択).
	\item Pen:加工データの線を色分けすることで,以下の項目を16パターン設定できる.
		\begin{itemize}
			\item Speed:加工速度(0.1\%~100.0\%の範囲で設定)
			\item Power:レーザーの出力(0\%~100\%の範囲で設定)
			\item PPI:1インチの線を描く間に何回レーザー光を発射するか(30~∞の範囲で設定)
		\end{itemize}
\end{itemize}

\section{ソースコード}
実際に使用したソースコードを以下に示す.


\end{document}

